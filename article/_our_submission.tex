%
% File acl2016.tex
%
%% Based on the style files for ACL-2015, with some improvements
%%  taken from the NAACL-2016 style
%% Based on the style files for ACL-2014, which were, in turn,
%% Based on the style files for ACL-2013, which were, in turn,
%% Based on the style files for ACL-2012, which were, in turn,
%% based on the style files for ACL-2011, which were, in turn, 
%% based on the style files for ACL-2010, which were, in turn, 
%% based on the style files for ACL-IJCNLP-2009, which were, in turn,
%% based on the style files for EACL-2009 and IJCNLP-2008...

%% Based on the style files for EACL 2006 by 
%%e.agirre@ehu.es or Sergi.Balari@uab.es
%% and that of ACL 08 by Joakim Nivre and Noah Smith

\documentclass[11pt]{article}
\usepackage{acl2016}
\usepackage{times}
\usepackage{url}
\usepackage{latexsym}
\usepackage{amsmath}
\usepackage{amssymb}
\usepackage{color,soul}
\usepackage[pdftex]{graphicx}
\usepackage{todonotes}

%\aclfinalcopy % Uncomment this line for the final submission
%\def\aclpaperid{***} %  Enter the acl Paper ID here

%\setlength\titlebox{5cm}
% You can expand the titlebox if you need extra space
% to show all the authors. Please do not make the titlebox
% smaller than 5cm (the original size); we will check this
% in the camera-ready version and ask you to change it back.

\newcommand\BibTeX{B{\sc ib}\TeX}
\newcommand\toin{\todo[inline]}

\title{Local Named Entity Disambiguation for Noisy Web Fragments with Neural Attention-RNNs}

\author{First Author \\
	Affiliation / Address line 1 \\
	Affiliation / Address line 2 \\
	Affiliation / Address line 3 \\
	{\tt email@domain} \\\And
	Second Author \\
	Affiliation / Address line 1 \\
	Affiliation / Address line 2 \\
	Affiliation / Address line 3 \\
	{\tt email@domain} \\}

\date{}

\begin{document}
\maketitle
\begin{abstract}
We address the task of Named Entity Disambiguation (NED) in a Web setting. We present WikilinksNED, a large-scale NED dataset of short web-page fragments containing mentions manually linked into Wikipedia by web content authors. In contrast to existing news-based datasets the web data is noisier and less coherent. We propose a model based on Attention-RNNs to capture useful features from short and noisy context. We evaluate our model on both WikilinksNED and a standard, smaller, news-based dataset. We find our model greatly outperforms existing state-of-the-art methods on WikilinksNED while achieving reasonable performance on the smaller dataset. We conclude our model can efficiently model noisy context given sufficient training. We analyze our results and show there is room for further improvement on such data.
\end{abstract}



\section{Introduction}

\toin{General comment about citations -- take whatever bibs you can from the ACL anthology: http://aclweb.org/anthology/

The bibs from Google scholar lack a lot of information.}

Named Entity Disambiguation (NED) is the task of linking mentions of entities in text to a given knowledge base, such as Freebase or Wikipedia. NED is a key component in Entity Linking (EL) systems, focusing on the disambiguation task itself, independently from the tasks Named Entity Recognition (detecting mention bounds) and Candidate Generation (retrieving the set of potential candidate entities). NED has been recognized as an important component in semantic parsing \cite{berant2013semantic}, as well as other NLP tasks.

NED algorithms can broadly be divided into local and global approaches. Local algorithms disambiguate each mention independently using local context (e.g. the sentence in which the mention appeared), whereas global approaches assume some coherence among mentions within a single document, and try to disambiguate all mentions simultaneously. Global algorithms have significantly outperformed the local approach on standard datasets \cite{guo2014entity,pershina2015personalized,globerson2016collective}. However, most of these datasets are based on news corpora and Wikipedia, which are naturally coherent, well-structured, and rich in context. Other domains, such as web page fragments, social media \cite{derczynski2015analysis}, or questions \cite{klang2014named}, lack the sufficient coherence and context for global models to pay off.\todo{Potentially give citations for each domain. YOTAM: not sure how to cite these. Is the QA cite good?} Take for example this fragment taken from the web:

\begin{quote}
	``I had no choice but to experiment with other indoor games. I was born in Atlantic City so the obvious next choice was \textbf{Monopoly}. I played until I became a successful Captain of Industry''
\end{quote}

This fragment is considerably less structured and with a more personal tone then news reports. It clearly references the entity \textit{Monopoly\_(Game)}, however expressions such as 'experiment', and 'Industry' can generate a lot of noise when disambiguating \textit{Monopoly\_(Game)} from the much more common entity \textit{Monopoly} (economics term). Some sense of local semantics and syntactics must be considered in order to separate the useful signals (e.g. indoor games, played) from the noisy ones.

In this work, we investigate the task of NED in a setting where only local and noisy context is available. In particular, we create a dataset of 3.2M short text fragments extracted from web pages, each containing a mention of a named entity. Our dataset contains 18K unique mentions linking to over 100K unique entities. This dataset is significantly larger than previously collected ones such as CoNLL-YAGO \cite{hoffart2011robust}, TAC KBP \cite{ji2010overview} and ACE 2010 \cite{bentivogli2010extending}. We have found that performance of state-of-the-art methods is greatly impaired on this dataset and believe new algorithms that can better model local semantic and syntactic features are required.

\todo{Why are you citing Hoffart 2011 if the task is from 2003? YOTAM: They were the ones to annotate it for NED}

We propose a novel neural network architecture based on Recurrent Neural Networks (RNNs) with an attention mechanism, where the RNN units model textual context as a sequence and the attention mechanism gives importance to contextual signals based on the specific candidate entity being evaluated. Our model differs from non-neural approaches by automatically learning feature representations for entity and context, allowing it to extract features from noisy and unexpected context patterns where it can be hard to manually design useful features. We differ from existing neural-based approaches by accounting for the sequential nature of textual context using RNNs and by devising an attention model that can reduce the impact of noise by assigning weights to different contextual signals based on the specific candidate entity being evaluated.

We also describe a novel method for initializing word and entity embeddings used in our model and demonstrate its importance for model performance and training efficiency. 

We demonstrate our model greatly outperforms existing state-of-the-art NED algorithms on our web based dataset, showing that existing state-of-the-art methods are not optimal in such settings, and that RNNs with attention can better model noisy and short context. In addition, we evaluate our algorithm on the CoNLL-YAGO dataset \cite{hoffart2011robust}, a dataset of annotated newswire articles, where it yields comparable performance to other state-of-the-art local methods. We conclude that RNNs with attention are well-suited for local disambiguation in clean and well-structured corpora, but that there is still much room for improvement in real world scenarios where text is short, noisy, and less coherent.

\section{Background}

\subsection{Related Work}

\toin{This section is a bit messy... Not sure I understand the flow. What are we trying to communicate here? Also, does it make the paper self-contained? The background should provide a reader who is not familiar with NED (but is from the general NLP/ML community) the necessary background information to be able to understand your contribution. Noam: Done}

In the past few years, the promising performance of global algorithms in Entity Linking (EL), Wikification and NED, established the domination of global approaches for the task of disambiguation. Nowadays, traditional and purely local solutions, such as those offered by \newcite{bunescu2006using} and \newcite{mihalcea2007wikify}, tend to be embedded as local context models in sophisticated collective global disambiguation systems. In this manner, the local component of the GLOW Wikification algorithm \cite{Ratinov2011} was exploited as part of the Relational inference system suggested by \newcite{Cheng2013}. Similarly, \newcite{Globerson2016} achieved state-of-the-art results by extending the local based selective-context model of \newcite{Lazic2015} with an attention-like coherence mechanism. 

Ablation analysis has shown more than once that coherency models boost the performance of local contextual baselines. However, it has also emphasized that the local techniques produce a hard baseline to beat \cite{Ratinov2011}. The coherence, which is mostly regarded as a property of the dataset, can also impose false constraints on the relatedness of entities in the document. For instance, \newcite{hoffart2011robust} had to build a coherency test on top of his collective graph-based model to disregard globally suggested Wikipedia entities. Accordingly, around $2/3$ of CoNLL's mentions where solved without even engaging the global coherence component. These observations, which were obtained on well structured news articles and Wikipedia based documents, might be amplified in a non-coherent textual environment.The fact that our suggested Wikilinks-based evaluation \cite{singh12:wiki-links} is crowd-sourced and much less coherent than traditional NED test-sets, led us to focus on generating a strong local context-based disambiguation solution.

\newcite{chisholm2015entity} incorporated web-link data from the same Wikilinks data source with Wikipedia to train a model for entity linking. Even though demonstrating superior results on newswire based evaluations, such as CoNLL, their study did not discuss its performance on the web-link mentions, which presents a very interesting NED challenge. Also, the textual context in that study is modeled in a Bag-Of-Word (BOW) fashion, which eliminates any compositional information of the text. \newcite{chisholm2015entity} \todo{Noam: lacks 's association} approach is very different from our suggested algorithm which allows capturing semantic information between context words by using a dedicated Deep Neural Network architecture (DNN). \todo{I didn't know that someone had already used Wikilinks for NED (although as training). We must state this explicitly, and emphasize the difference/contribution of your work. Noam: Done}.

The first published attempt of using DNN for NED was led by \newcite{He2013}, in which the network learned a similarity measure between mention-context structures and candidates from Wikipedia using stacked autoencoders. Recent studies, have suggested Convolutional Neural Nets (CNN) architectures for learning semantic similarity between all three context, mention and candidate inputs \cite{sun2015modeling,francis2016capturing}. The growing popularity of neural embeddings in NLP related tasks has inspired several researches to jointly map those inputs to the same space using the fantastic word2vec approach \cite{yamada2016joint,Melamud2014}.

In this paper, we embed both words and Wikipedia entities in the same space to form input vectors for a Recurrent Neural Network (RNN) model. Unlike other studies, we not only exploit the sequential structure of the local surrounding context, but also incorporate semantic world knowledge from a much larger corpus in our representation of text. Moreover, following the results of \newcite{Lazic2015}, who claimed that only few context words have value in disambiguating the mention, we equipped our model with an attention module, which significantly reduces the effect of non-informative neighboring text. Our novel solution combines all former properties to produce state-of-the-art local NED results on the suggested Wikilinks data set.

\subsection{Datasets for NED}

One of the most commonly used evaluations for benchmarking the challenge of NED \cite{Globerson2016,Hachey2013,yamada2016joint,Pershina2015} is the CoNLL corpus, which was crafted from the CoNLL 2003 Named Entity Recognition (NER) task. This evaluation was established by \newcite{hoffart2011robust}, which manually annotated Reuters newswire articles from 1996 with corresponding entities in the YAGO knowledge base. Its contains $1393$ documents from a period of $12$ days split into train, development and test sets. Following previous works we have only evaluated our method on non-NIL mentions.\todo{How was it annotated? By experts? Which KB was it mapped to? Noam: done}

TAC KBP \cite{ji2010overview} is another popular dataset for testing the performance of disambiguation systems \cite{chisholm2015entity,Globerson2016,sun2015modeling}. Similar to CoNLL, it is primarily based on news articles and was specifically designed for the task of EL in the Text Analysis Conference (TAC). The most broadly used version for NED is the TAC 2010 data set, which includes a training and test set of $1,070$ and $1,017$ non-NIL mentions, respectively \footnote{This data set was not available to us, as it is distributed only to TAC participants.}.

Some disambiguation systems are also evaluated on the ACE 2005 corpus \cite{Ratinov2011,francis2016capturing}. Being mainly composed out of news reports from various sources, this corpus was extended by connecting and annotating it using Wikipedia \cite{bentivogli2010extending}. It features $16,310$ annotated mentions when only $1,458$ of them have multiple links, hence making them challenging for NED. A different type of evaluation, suggested by \newcite{Ratinov2011}, is completely composed out of paragraphs from Wikipdia pages. In this Wiki dataset, cross-reference hyperlinks act as mention's surface, thus implying that data is edited in a more crowd-sourced fashion.

All aforementioned datasets share the property of well formed and structured content, since they are founded on the publications of recognized information and media organizations. However, when focusing only on this type of data we overlook the fact that today most information is transfered without any filtering via a variety of sources, such as blogs, social network posts and group chats. These sources incorporate significantly higher contextual noise as they lack proper redaction. Additionally, most of traditional NED evaluations were manually annotated and therefore contain a very small training set for supervised disambiguation techniques. In this study we devise a novel benchmark of crowd source text fragments that were manually linked to their corresponding Wikipedia titles by their authors. This relatively incoherent source offers a vast and natural corpus for supervised algorithms and produces an interesting case study for evaluating NED on more general and common text.

\toin{חבר׳ה, יש פה הרמה להנחתה שאתם לא מנצלים... מה רע בדאטאסטים האלה? מה אתם מחדשים על פניהם? Noam: fixed}

\toin{This section provides readers who are less familiar with the literature the necessary information to understand your contribution. Things that need to appear in this section:
	
	- Previous work on NED.
	
	- An in-depth survey of the existing datasets and how they were built.
	
	- Neural work on NED.
	
	At the end of each paragraph/subsection, mention how this work improves upon / differs from what you just discussed.
}

\section{WikilinksNED Dataset: Entity Mentions in the Web}
\label{sec:w}

We introduce a new large-scale NED dataset based on text fragments from the web. Our dataset is derived from the Wikilinks corpus \cite{singh12:wiki-links}, which was constructed by crawling the web and collecting hyperlinks (mentions) linking to Wikipedia concepts (entities) and their surrounding text (context). Wikilinks contains 40 million mentions covering 3 million entities, collected from over 10 million web pages. 

Wikilinks can be seen as a large-scale, naturally-occurring, crowd-sourced dataset where thousands of human annotators provide ground truths for mentions of interest. This means that the dataset also contains various kinds of noise, including erroneous ground-truth labels, malformed mentions, and incoherent contexts. The contextual noise in particular presents an interesting test-case that supplements existing datasets such as CoNLL-YAGO \cite{hoffart2011robust}, TAC KBP \cite{ji2010overview} and ACE 2010 \cite{bentivogli2010extending}, since these datasets are all sourced from mostly coherent and well-formed text (news and Wikipedia). Wikilinks therefore emphasizes the need to understand the local context, and marginalizes the utility of coherency-based global approaches. 

To get a sense of textual noise and entity-context coherence we have set up a small experiment where we measured the similarity between entities mentioned in WikilinksNED and their surrounding context, and compared the results to CoNLL-YAGO. We used state-of-the-art word and entity embeddings obtained from Yamada at el \cite{yamada2016joint} and computed cosine similarity between an entity embedding and the mean of context words embeddings. We compared results from all mentions in CoNLL-YAGO to a sample of 10000 web fragments taken from WikilinksNED, using a window of words of size 40 around entity mentions. On CoNLL-YAGO we found the mean similarity to be 0.188 while on WikilinksNED we got 0.163. We believe this result indicates that web fragments in WikilinksNED are indeed less coherent and noisier compared to CoNLL-YAGO documents.

\todo{I agree with the reasoning, but we should have some sort of experiment/analysis to back this claim. YOTAM: what about this?}

We prepared our dataset from the local-context version of Wikilinks,\footnote{\url{http://www.iesl.cs.umass.edu/data/wiki-links}} and resolved ground-truth links from the 7/4/2016\todo{American or international dating? YOTAM: International... How can we make this clear?} dump of Wikipedia \footnote{\url{https://dumps.wikimedia.org/}}. We used the \emph{page} and \emph{redirect} tables for resolution, and kept the database \emph{pageid} column as a unique identifier for Wikipedia entities. We discarded mentions where the ground-truth could not be resolved (only 3\% of mentions).

We collected all pairs of mention $m$ and entity $e$ appearing in the dataset, and computed the number of times $m$ refers to $e$ ($\#(m,e)$), as well as the conditional probability of $e$ given $m$: $P(e|m)=\#(m,e)/\sum_{e'}\#(m,e')$. Examining these distributions revealed many mentions belong to two extremes -- either they had very little ambiguity, or they appeared in the dataset only a handful of times and referred to different entities only a couple of times each.\todo{Not sure I understand the second case. Example? YOTAM: tried to improve explanation. } We deemed the former to be less interesting for the purpose of NED, and suspected the latter to be noise with high probability. To filter these cases, we kept only mentions for which at least two different entities have 10 mentions each ($\#(m,e) \ge 10$) and consist of at least 10\% of occurrences $P(e|m) \ge 0.1$. This procedure aggressively filtered our dataset and we were left with $3.2M$ mentions.

Finally, we randomly split the data into train (90\%), validation (10\%), and test (10\%), according to domains in order to minimize lexical memorization (see \cite{levy2015supervised}).


\section{Algorithm}

\toin{This section should be organized as follows: 4. Overview, 4.1. Model Architecture, 4.2. Training, 4.3. Embedding Initialization. Overview should refer to the diagram, which 4.1. elaborates. 4.1. should include all the formulae, as well as the rationale behind the architecture.}

Our DNN model is a discriminative model which takes a pair of local context and candidate entity, and outputs a likelihood for the candidate entity being correct.\todo{Omer, check this again. Is it a factor or a probability?}. Both words and entities are represented using embedding dictionaries and we interpret local context as a window-of-words to the left and right of a mention. The left and right contexts are fed into a duo of Attentional RNNs (ARRN) components which process each side and produce a fixed length vector representation. The resulting vectors along with the entity embedding are then fed into a classifier network with two output units that are trained to emit the likelihood of the candidate being a correct or corrupt assignment. 

\subsection{Model Architecture}

\begin{figure}
	\centering
	\includegraphics[scale=0.25]{diagrams/RNN_ATTN.pdf}
	\caption{Attentional RNN Architecture}
	\label{fig:arnn}
\end{figure}	
  
Our architecture has two main component: a duo of ARNN components and a classifier. The classifier network consists of a hidden layer\footnote{300 dimensions with ReLU, and $p=0.5$ dropout.} and an output layer with two output units in a softmax. The output units are trained to emit the likelihood of the candidate being a correct or corrupt assignment by optimizing a cross-entropy loss function. 

The ARNN components each process one side of the context (left and right) where the right context is fed into the ARNN in reverse. The ARNNs are based on a general RNN unit fitted with an attention mechanism, and their mechanics are depicted in Figure \ref{fig:arnn}. 

Equation \ref{eq1} represents the general semantics of an RNN unit. An RNN reads a sequence of vectors $\{v_t\}$ and maintains a hidden state vector $\{h_t\}$. At each step a new hidden state is computed based on the previous hidden state and the next input vector by a function $f$ parametrized by $\Theta_1$. The output at each step is computed from the hidden state using a function $g$ parametrized by $\Theta_2$. This allows the RNN to 'remember' important signals while scanning the context and to recognize signals spanning multiple words.

\begin{equation}
\label{eq1}
\begin{aligned}
& h_t=f_{\Theta_1}(h_{t-1}, v_t) \\
& o_t=g_{\Theta_2}(h_t)
\end{aligned}
\end{equation}

In our implementation we have used a standard GRU unit \cite{cho2014learning}, however any RNN can be a drop-in replacement. We fit the RNN unit with an additional attention mechanism, a mechanism commonly used for state-of-the-art encoder-decoder models where the output of the decoder at time step $t$ is used to decide which parts of the encoder output are important for decoding time step $t+1$ \cite{bahdanau2014neural, xu2015show}. Since our model lacks a decoder, we use the entity embedding as a control signal for the attention mechanism.

Equation \ref{eq2} details the equations governing the attention model.

\begin{equation}
\label{eq2}
\begin{aligned}
& a_t \in \mathbb{R}; a_t=r_{\Theta_3}(o_t, v_{candidate}) \\
& a'_t  = \frac{1}{\sum_{i=1}^{t} \exp\{a_i\}} \exp \{a_t\} \\
& o_{attn}=\sum_{i=1}^{t} a'_t o_t
\end{aligned}
\end{equation}

The main component in equation \ref{eq2} is the function $r$, parametrized by $\Theta_3$, which computes an attention value at each step using $v_{candidate}$, the candidate entity embedding, as a control signal. A softmax function then normalizes the attention values and the final output $o_{attn}$ is computed as a weighted sum of all the output vectors of the RNN. This allows the attention mechanism to decide on the importance of different context parts when examining a specific candidate. We follow to Bahdanau at el. \cite{bahdanau2014neural} and parametrize the attention function $r$ is pas a single layer NN as shown in equation \ref{eq3} where $A, B$ are the layer weights and $b$ is a bias term.

\begin{equation}
\label{eq3}
r_{\Theta_3}(o_t, v_{candidate}) = Ao_t + Bv_{candidate} + b \\
\end{equation}

\toin{Is this the first-ever use of attention in GRUs, or was this parametrization used before? It reads as if your contribution is inventing this mechanism, rather than using it. If this is not novel, I suggest rephrasing and citing appropriately; the explanation is very clear, but it needs to emphasize what you did *differently*.}

\subsection{Training}

We assume our model is only given examples of correct entity assignments during training and therefore automatically generate examples of corrupt assignments. For each context-entity pair $(c,e)$, where $e$ is the correct assignment for $c$, we produce $k$ corrupt examples with the same context $c$ but with a different, corrupt entity $e'$, which is sampled uniformly from all entities in the dataset. Using the combined dataset of correct and corrupt examples, our algorithm learns to separate correct assignments from the generated corrupt ones.

Sampling corrupt assignments from the set of all entities rather then only possible candidates for each mention was done for two reasons: it diversifies the contexts in which each entity is given to the model as a negative example allowing the model to better generalize for unseen mentions and contexts, and allows to utilize unambiguous examples where only a single candidate is found (i.e. Wikipedia cross-references used in section \ref{experiments-conll} which contain a large number of unambiguous mentions) \todo{YOTAM: is this explanation good?}

Uniform sampling is done to match the distribution of positive examples of an entity to the prior probability of the entity in the corpus - in effect biasing the network towards more frequent entities. \todo{YOTAM - should i add a formula to show this? i can show this with a formula but its a bit tedious}

Model optimization was carried out using standard backpropagation and an AdaGrad optimizer \cite{duchi2011adaptive}. We allowed the error to propagate through all parts of the network and fine tune all trainable parameters, including the word and entity embeddings themselves. In our experiments we trained all models for $20$ epochs and used $k=5$ for corrupt example generation. We found the performance of our model substantially improves for the first few epochs and then continues to slowly converge with marginal gains. 

\todo{YOTAM - I have the training graph, here or in the evaluation section? }

\subsection{Embedding Initialization}

\toin{In the model architecture, and perhaps also in the overview, you should mention that both words and entities are embedded. In the training subsection, you should mention whether the embeddings were finetuned or not.}
Training our model implicitly embeds the vocabulary of words and collection of entities in a common space. However, we find that explicitly initializing these embeddings with vectors pre-trained over a large collection of unlabeled data significantly improved \hl{whatever} (see Section \ref{experiments}\todo{Refer to the exact subsection where you discuss this result}). To this end, we implemented an SGNS-based approach \cite{mikolov2013distributed} that simultaneously trains both word and entity vectors.

We used \texttt{word2vecf}\footnote{\url{http://bitbucket.org/yoavgo/word2vecf}} \cite{levy2014dependency}, which allows one to train word and context embeddings using arbitrary definitions of "word" and "context" by providing a dataset of word-context pairs $(w,c)$, rather than a textual corpus. In our usage, we define a context as an entity $e$. To compile a dataset of $(w,e)$ pairs, we consider every word $w$ that appeared in the Wikipedia article describing entity $e$. We limit our vocabularies to words that appeared at least 20 times in the corpus and entities that contain at least 20 words in their articles. We ran the process for 10 epochs and produced vectors of 300 dimensions; other hyperparameters were set to their defaults.

\newcite{levy2014neural} showed that SGNS implicitly factorizes the word-context PMI matrix. Our approach is doing the same for the word-entity PMI matrix, which is highly related to the word-entity TFIDF matrix used in Explicit Semantic Analysis \cite{gabrilovich2007computing}.

\hl{-needs developing-}

\hl{-show results of the analogies experiment we did indicating semantic structure for the WORD vectors-}


\section{Evaluation}
\label{experiments}

In this section, we describe our experimental setup and compare our model to the state of the art on two datasets: our new WikilinksNED dataset, as well as the commonly-used CoNLL-YAGO dataset \cite{hoffart2011robust}. We also examine the effect of initializing our model with pre-trained word and entity embeddings.

In all experiments, our model was trained with fixed-size left and right contexts (20 words in each direction). We used a special padding symbol when the actual context was shorter than the window. Further, we filtered stopwords using NLTK's stop-word list prior to selecting the window in order to focus on more informative words.

We experiment with our model both as a standalone model and as a feature along with shallow statistical and string based features. For the second case, we used a setting similar to Yamada et al. \shortcite{yamada2016joint} where a Gradient Boosted Regression Tree was fitted with our models prediction score as a feature along with $7$ other statistical and string based features. The statistical features are prior probability $P(e)$ and conditional probability $P(e|m)$ as described above, along with a feature counting the number of candidates generated for the mention and a feature giving the maximum conditional probability of the entity for all mentions in the document. For string similarity features we used the edit distance between the mention and the entity title in Wikipedia, a feature indicating weather the mention is a prefix or postfix of the entity Wikipedia title and a feature indicating weather the Wikipedia entity title is a prefix or postfix of the mention. Following Yamada we used sklearn's GradientBoostingClassifier implementation \cite{pedregosa2011scikit} with a deviance loss and set the learning rate, number of estimator and maximum depth of a tree to $0.02$, $10000$ and $4$, respectively. 

\subsection{WikilinksNED}

\paragraph{Candidate Generation}
To isolate the effect of candidate generation algorithms, we used the following simple method for all systems: given a mention $m$, consider all candidate entities $e$ that appeared as the ground-truth entity for $m$ at least once in the training corpus. This simple method yields $97\%$ ground-truth recall on the test set.

\paragraph{Baselines}
Since we are the first to evaluate NED algorithms on WikilinksNED, we ran a selection of existing local NED systems and compared their performance to our algorithm's. \textbf{Yamada et al.} \shortcite{Yamada2016} created a state-of-the-art NED system that models entity-context similarity with word and entity embeddings trained using the skip-gram model. We obtained the original embeddings from the authors, and trained the statistical features and ranking model on the WikilinksNED training set. Our configuration of Yamada et al.'s model used only their local features.

\textbf{Cheng et al.} \shortcite{Cheng2013} have made their global NED system publicly available \footnote{\url{https://cogcomp.cs.illinois.edu/page/software\_view/Wikifier}}. This algorithm uses GLOW \cite{Ratinov2011} for local disambiguation. We compare our results to the ranking step of the algorithm, without the their global component. Due to the long running time of this system, we only evaluated their method on the smaller test set, which contains 10,000 randomly sampled instances from the full 320,000-example test set.

Finally, we include the \textbf{Most Probable Sense (MPS)} baseline, which selects the entity that was seen most with the given mention during training.

\begin{table*}[t]
	\begin{center}
		\begin{tabular}{|c| c | c | }
			\hline \multicolumn{3}{|c|}{Wikilinks Evaluation} \\
			\hline \bf Model               & \bf Sampled Test Set (10K)  & \bf Full Test Set (300K)  \\
			Baseline (MPS)                 & $60$   & $59.6$ \\
			Cheng et al.                   & $50.7$ & - \\
			Yamada at el.                  & $67.6$ & $66.9$ \\
			\hline
			Our ARNN                       & $70.8$ & $70.4$ \\
			\bf ARNN + shallow features    & ?      &  $?$ \\
			\hline
		\end{tabular}
	\end{center}
	\caption{\label{tab:wikilink} Evaluation on Web-Fragment data (Wikilinks)}
\end{table*}


\paragraph{Results}
Experimental results on the Wikilinks dataset are reported in Table \ref{tab:wikilink}. Our algorithm significantly outperforms Yamada at el. on this data by a substantial margin. This result indicates that the skip-gram model used by Yamada at el. which averages the embedding vectors of all context words is non-optimal compared to our more sophisticated context modeling on this dataset. Our method outperforms the Baseline as well by a very large margin indicating our RNN model is indeed able to capture meaningful contextual features despite the noisy and short context. 

We find that supplementing our model with standard statistical and string based features improves performance. This can be expected since input to our model does not include any string or statistical features. 

When running Cheng at el \shortcite{Cheng2013} we have used a pre-trained model supplied by Cheng at el which, similarly to the setting used for evaluating the GLOW algorithm by Ratinov at el \cite{ratinov2011glow}, was not directly trained on our training set. This has resulted in poor performance, emphasizing the greater importance of training a model directly on the training set compared to existing datasets based on news corpora and annotated by experts.

\subsection{CoNLL-YAGO}
\label{experiments-conll}

\paragraph{Training}
CoNLL-YAGO has a training set with $18505$ non-NIL mentions, which preliminary experiments showed is not sufficient to train our model on. We therefore resorted to a more complex training method where we first trained our model on a large corpus of mentions derived from Wikipedia cross-references and then fine-tuned the resulting model on CoNLL-YAGO training set. To derive the Wikipedia training corpus we have extracted all cross-reference links from Wikipedia along with their context, resulting in over $80$ million training examples. We set $k=5$ for corrupt example generation and trained for $1$ epoch. The resulting model was then fine-tuned on CoNLL-YAGO training set, where corrupt examples were produced by considering all possible candidates for each mention.

\paragraph{Candidate Generation}
For comparability with existing methods we used two publicly available candidates datasets:
\begin{itemize}
	\item PPRforNED - Pershina at el. \shortcite{Pershina2015}
	\item YAGO - Hoffart at el. \shortcite{hoffart2011robust}
\end{itemize}

\paragraph{Baselines}
As a baseline we took the standard Most Probable Sense (MPS) prediction, which corresponds to the $\arg\max_{e\in{{E}}}{P(e|m)}$, where $E$ is the group of all candidate entities.
We also compare to the following papers - Francis-Landau et al. \shortcite{francis2016capturing}, He et al. \shortcite{He2013}, Hoffart et al. \shortcite{hoffart2011robust} and Chisholm et al. \shortcite{Chisholm2015} ,as they are all strong local approaches and a good source for comparison.

\paragraph{Results}
The micro and macro P@1 scores on CoNLL-YAGO test-b are displayed in table \ref{tab:conll}. On this dataset our model achieves reasonable results, however it cannot beat state-of-the-art results since it requires large quantities of training data to properly train the large number of parameters in the model.  Further, the relative cleanliness of the data allows existing approaches to perform very well and marginalizes the effect of utilizing a deep and powerful neural model.

\begin{table}[h]
	\begin{center}
		\begin{tabular}{|p{3.5cm}| p{1.5cm} p{1.5cm}|}
			\hline \multicolumn{3}{|c|}{CoNLL-YAGO test-b (Local methods)} \\
			\hline Model & Micro P@1 & Macro P@1 \\ 
			\hline \multicolumn{3}{|c|}{PPRforNED} \\
			\hline ARNN + Shallow features    & $87.3$  & $88.6$ \\
			Yamada et al. local               & $90.9$  & $92.4$ \\
			\hline \multicolumn{3}{|c|}{Yago} \\
			\hline ARNN + Shallow features    & $83.3$  & $86.3$ \\
			Yamada et al. local               & $87.2$  & $89.6$ \\
			Francis-Landau et al.             & $85.5$  & - \\
			Chisholm et al. local             & $86.1$  & - \\
			\hline Lazic et al.               & $86.4$  & - \\
			\hline
		\end{tabular}
	\end{center}
	\caption{\label{tab:conll} Evaluation on CoNLL-YAGO.}
\end{table}

\subsection{Effects of initialized embeddings and attention}

We performed a study of the effects of the main components of our system. In particular we have evaluated the effects of using an attention mechanism and of using pre-initialized word and entity embeddings. The evaluation can be seen in Table \ref{tab:c}. We found that the attention model indeed improves the performance of our model by 0.7 points, a small yet significant margin. (???) We have also found that using pre-initialized embeddings greatly speeds up convergence of the model and ultimately leads to slightly better performance (???)

\begin{table}[h]
	\begin{center}
		\begin{tabular}{|c| p{1.5cm}|}
			\hline \multicolumn{2}{|c|}{Wikilinks test set} \\
			\hline \bf Model & \bf Micro     accuracy  \\ \hline
			RNN w/o Attention & $69.7$ \\
			ARNN w/o Initialization & $?$ \\
			ARNN & $70.4$ \\ 
			\hline
		\end{tabular}
	\end{center}
	\caption{\label{tab:c} ARNN Model sensetivity}
\end{table}

\subsection{Discussion//Conclusions}
We believe these results demonstrate that our ARNN model is better then existing state-of-the-art techniques at modeling noisy context when sufficient amounts of training examples are available. This allows us to greatly outperform state-of-the-art algorithms which are found to be suboptimal in a noisy environment such as web content. 

However the gap between results of all systems tested on both CoNLL-YAGO and WikilinksNED indicates that web mentions are indeed a more challenging test case. We believe recursive neural models are a promising direction for this tast however there is still room for improvement on this problem. 

- wait for error analysis (+ entities lacking sufficient training data? ) -

\subsection{Error analysis}

%We extracted and analyzed $4227$ cases of false disambiguated mentions that our ARNN model predicted. This subset was obtained from evaluating on a Wikilinks-based validation set that was not used for training. 

%While going over the errors, we came across several examples where the actual entity was fully contained in the prediction. For instance, gold-senses such as \textit{'ted'} and \textit{'macedonia'} were "falsify confused" with the entities \textit{'ted (conference)'} and \textit{'republic of macedonia'}. In practice, the context of those examples together with their correct urls links to the same source as the models prediction. This type of error was observed in almost $20$\% of the cases. We also saw that in these $20$\% the conditional prior of the predicted entity was larger than the conditional of the correct sense. We believe that this indicates on the generic nature of the model and emphasizes problems in the candidate generation process.

%A further interesting property of the errors was the distance between the false predicted entity and the correct one. This was measured by calculating the cosine similarity between our word-title embeddings. For example, the similarity between \textit{'the fall (gorillaz album)'} and \textit{'the fall (band)'} was calculated as $0.96$. Accordingly, given that music related words, such as 'recorded' and 'albums', were present in the context of the mention, the decision to pick the wrong sense over the right one does not seem completely detached and irrational. In our analysis the average cosine similarity of the mistakes was estimated to be $0.925$, when $0.9$ was the $20$st percentile of the value distribution. Strong associations between candidates are usually expected in a disambiguation framework, since most possible senses are somewhat related. However, given that the embeddings use as a main feature for our model, we believe that the particularly high error similarity shows that even the wrong predicted entities are strongly related to right senses.    \toin{fix and report with Yamadas embeddings} 

%Lastly, we wanted to check how much our model differs from the MPS baseline by visualizing the error distribution of the conditional prior feature. In this way we could quantify the diversity of the model's prediction. The full error distribution is displayed in. We saw that $64$\% of the error predictions were not most frequent entities. the average conditional prior of those non-mps mistakes was $0.197$\% with $0.664$ coefficient of variation.  Moreover, we observed that $25$st percentile of the error was smaller than $0.09$\%. This error analysis reveals that in most of its mistakes our model does not simply captures [continue]

\bibliographystyle{acl2016}
\bibliography{_our_submission,_noam_bib}

\end{document}
